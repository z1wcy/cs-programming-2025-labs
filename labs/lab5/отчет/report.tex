\documentclass{vvsu}

\vvsuyear{2025}

%%%%%%%%%%%%%%%%%%%

\usepackage{graphicx} % для изображений
\usepackage{tabularray} % для таблиц
\usepackage{siunitx} % для обозначений (процент, градус)
\usepackage{csquotes} % для корректной работы с кавычками
\usepackage{listings} % для листингов кода

% Список путей, где будут искаться изображения и файлы
\graphicspath{{images/}}

% Автор документа
\author{М. В. Кирийчук}

% Настройка стилей для листингов кода
\setmonofont{Consolas}

\makeatletter

\newcommand\language@yaml{yaml}
\expandafter\expandafter\expandafter\lstdefinelanguage
\expandafter{\language@yaml}
{
  keywords={true,false,null,y,n},
  keywordstyle=\color{darkgray},
  basicstyle=\setmainfont{Consolas}\fontsize{8}{8}\linespread{1}\selectfont,
  sensitive=false,
  comment=[l]{\#},
  morecomment=[s]{/*}{*/},
  commentstyle=\color{purple},
  stringstyle=\color{blue},
  moredelim=[l][\color{orange}]{\&},
  moredelim=[l][\color{magenta}]{*},
  moredelim=**[il][\color{red}{:}\color{blue}]{:},
  morestring=[b]',
  morestring=[b]",
  literate =    {---}{{\ProcessThreeDashes}}3
                {>}{{\textcolor{red}\textgreater}}1
                {|}{{\textcolor{red}\textbar}}1
                {\ -\ }{{\ -\ }}3,
}
\lst@AddToHook{EveryLine}{\ifx\lst@language\language@yaml\color{black}\fi}
\makeatother

\lstdefinelanguage{json}{
    basicstyle=\fontsize{8}{8}\linespread{1}\selectfont\ttfamily,
    sensitive=false,
    stringstyle=\color{blue},
    string=[s]{":\ "}{"},
    literate=
        *{0}{{{\color{red}0}}}{1}
         {1}{{{\color{red}1}}}{1}
         {2}{{{\color{red}2}}}{1}
         {3}{{{\color{red}3}}}{1}
         {4}{{{\color{red}4}}}{1}
         {5}{{{\color{red}5}}}{1}
         {6}{{{\color{red}6}}}{1}
         {7}{{{\color{red}7}}}{1}
         {8}{{{\color{red}8}}}{1}
         {9}{{{\color{red}9}}}{1}
}

\definecolor{codegray}{rgb}{0.5,0.5,0.5}
\definecolor{backcolour}{rgb}{0.95,0.95,0.92}
\lstdefinestyle{codestylelst}{
    backgroundcolor=\color{backcolour},
    numberstyle=\color{codegray}\ttfamily,
    breakatwhitespace=false,
    breaklines=true,
    captionpos=b,
    keepspaces=true,
    numbers=left,
    numbersep=5pt,
    showspaces=false,
    showstringspaces=false,
    showtabs=false,
    tabsize=2
}
\lstset{style=codestylelst}


%%%%%%%%%%%%%%%%%%%

\begin{document}

% Шапка
\vvsuhead{\linespread{1}\selectfont{}МИНОБРНАУКИ РОССИИ\\
\vspace{10pt}Федеральное государственное бюджетное образовательное учреждение\\
высшего образования\\
\fontsize{13}{13}\selectfont{}<<ВЛАДИВОСТОКСКИЙ ГОСУДАРСТВЕННЫЙ УНИВЕРСИТЕТ>>\\
(ФГБОУ ВО <<ВВГУ>>)\\
\vspace{10pt}\fontsize{12}{12}\selectfont{}ИНСТИТУТ ИНФОРМАЦИОННЫХ ТЕХНОЛОГИЙ И АНАЛИЗА ДАННЫХ\\
КАФЕДРА ИНФОРМАЦИОННЫХ ТЕХНОЛОГИЙ И СИСТЕМ}

% Название отчета
\title{Отчет\\по лабораторной работе №5}
\subtitle{по дисциплине\\<<Информатика и программирование>>}

% Участники работы
\member{Студент\\ гр. БИН-25-3}{М.В. Кирийчук}
\member{Ассистент\\ преподавателя}{М.В. Водяницкий}

% Вывод титульника
\maketitle

% Задание
\begin{addition}{Задание}
  Выполнить задания и оформить отчет по стандартам ВВГУ.

  \textit{\textbf{Задание 1.}}
  Дан список из 10 различных целых чисел. Необходимо найти в нем число 3 и заменить на 30.

  \textit{\textbf{Задание 2.}}  
  Дан список из 5 целых чисел. Необходимо превратить его в список квадратов этих чисел.

  \textit{\textbf{Задание 3.}}  
  Имеется список различных целых чисел. Программа должна найти наибольшее из чисел списка и разделить его на длину списка.

  \textit{\textbf{Задание 4.}}  
  Имеется кортеж из нескольких произвольных элементов. Необходимо этот кортеж отсортировать. Если хотя бы один элемент не является числом, то кортеж остается неизменным.

  \textit{\textbf{Задание 5.}}  
  Имеется словарь товаров в магазине. Необходимо найти товар с минимальной и максимальной ценой.

  \textit{\textbf{Задание 6.}}  
  Имеется список произвольных элементов. Необходимо на основе этого списка создать словарь, где каждый элемент списка будет и ключом, и значением.
  
  \textit{\textbf{Задание 7.}}  
  Имеется словарь перевода английских слов на русский, где ключ английского слово, значение - русского. Необходимо реализовать программу которая получает на ввод русское слово и результатом выдает перевод на английский.

  \textit{\textbf{Задание 8.}}  
  Реализовать игру Камень-Ножницы-Бумага-Ящерица-Спок. Программа должна запрашивать у пользователя ввод одного из вариантов. Второй вариант случайно генерирует сама программа и возвращает победителя.

  Правила игры следующие:
\begin{vvsu_itemize}
    \item Ножницы режут бумагу
    \item Бумага покрывает камень
    \item Камень давит ящерицу
    \item Ящерица отравляет Спока
    \item Спок ломает ножницы
    \item Ножницы обезглавливают ящерицу
    \item Ящерица съедает бумагу
    \item Бумага подставляет Спока
    \item Спок испаряет камень
    \item Камень разбивает ножницы
\end{vvsu_itemize}

  \textit{\textbf{Задание 9.}}  
  Дан список слов - например:
  \begin{vvsu_itemize}
    \item `["яблоко", "груша", "банан", "киви", "апельсин", "ананас"]`
  \end{vvsu_itemize}

  Необходимо создать новый словарь, где:

  \begin{vvsu_itemize}
    \item Ключом будет первая буква слова
    \item Значением - список всех слов, начинающихся с этой буквы
  \end{vvsu_itemize}
  
  Пример результата:

    {'я': ['яблоко'], 'г': ['груша'], 'б': ['банан'], 'к': ['киви'], 'а': ['апельсин', 'ананас']}
  
  \textit{\textbf{Задание 10.}}  
  Дан список кортежей, где каждый кортеж содержит имя студента и его оценки, например:
  
  [("Анна", [5, 4, 5]), ("Иван", [3, 4, 4]), ("Мария", [5, 5, 5])]

  Необходимо:

  \begin{vvsu_itemize}
    \item Создать словарь, где ключ - имя студента, значение - средняя оценка
    \item Найти студента с наивысшей средней оценкой
  \end{vvsu_itemize}
\end{addition}

% Содержание
\toc

% Глава - Выполнение работы
\section{Выполнение работы}
\label{sec:execution}

% Подглава - Задание 1
\subsection{Задание 1}
  Сначала создадим список lists, заменяем число 3 на 30, а после выводим результат. На рисунке 1 представлен код программы.

\begin{vvsu_figure}{Листинг программы для задания 1}{labs\lab5\lab5.1.py}
  \begin{minipage}{.75\textwidth}
    \lstinputlisting[language=Python,basicstyle=\fontsize{10}{10}\linespread{1}\selectfont\ttfamily]{labs\lab5\lab5.1.py}
  \end{minipage}
\end{vvsu_figure}

\begin{vvsu_list}
  \item создадим спикок из 10 различных целых чисел;
  \item ищем в списке число 3 и заменяем его на 30;
  \item выводим результат
\end{vvsu_list}

% Подглава - Задание 2
\subsection{Задание 2}
 Создаем список из 5 различных чисел, а дальше нам нужно превратить его в список квадратов этих чисел. На рисунке 2 представлен код программы.

\begin{vvsu_figure}{Листинг программы для задания 2}{labs\lab5\lab5.2.py}
  \begin{minipage}{.75\textwidth}
    \lstinputlisting[language=Python,basicstyle=\fontsize{10}{10}\linespread{1}\selectfont\ttfamily]{labs\lab5\lab5.2.py}
  \end{minipage}
\end{vvsu_figure}

\begin{vvsu_list}
  \item создаем список из 5 различных чисел;
  \item используем генератор списков для возведения каждого числа в квадрат и выводим результат.
\end{vvsu_list}

% Подглава - Задание 3
\subsection{Задание 3}
  Нам требуется создать произвольный список числе, а далее программа должна найти наибольшее из чисел списка и разделить его на длину списка. После выводим в консоль результат. На рисунке 3 представлен код программы.

\begin{vvsu_figure}{Листинг программы для задания 3}{labs\lab5\lab5.3.py}
  \begin{minipage}{.75\textwidth}
    \lstinputlisting[language=Python,basicstyle=\fontsize{10}{10}\linespread{1}\selectfont\ttfamily]{labs\lab5\lab5.3.py}
  \end{minipage}
\end{vvsu_figure}

\begin{vvsu_list}
  \item создаем произвольный список числе;
  \item находим максимальное число из списка, далее делим на длину списка и выводим результат пользователю.
\end{vvsu_list}

% Подглава - Задание 4
\subsection{Задание 4}
  Создаем функцию которая проверяет состоит ли кортеж только из чисел, если да, то сортируем числа от еньшего к большему, если неь, то кортеж остается неизменным. На рисунке 4 представлен код решения.

\begin{vvsu_figure}{Листинг программы для задания 4}{labs\lab5\lab5.4.py}
  \begin{minipage}{.75\textwidth}
    \lstinputlisting[language=Python,basicstyle=\fontsize{10}{10}\linespread{1}\selectfont\ttfamily]{labs\lab5\lab5.4.py}
  \end{minipage}
\end{vvsu_figure}

\begin{vvsu_list}
  \item объявляем функцию sort с параметром х;
  \item проверка, все ли элементы в x - числа (int или float);
  \item если возвращает значение True, то сортируем кортеж и возвращаем отсортированный кортеж;
  \item если возвращает значение False, то возвращаем кортеж без изменений;
  \item проверка работы функции значением True;
  \item проверка работы функции с значением False.
\end{vvsu_list}

% Подглава - Задание 5
\subsection{Задание 5}
  Создае функцию которая получает на ввод словарь и выведет пользователю товар с максимальной и минимальной ценой. На рисунке 5 представлен код программы.

\begin{vvsu_figure}{Листинг программы для задания 5}{labs\lab5\lab5.5.py}
  \begin{minipage}{.75\textwidth}
    \lstinputlisting[language=Python,basicstyle=\fontsize{10}{10}\linespread{1}\selectfont\ttfamily]{labs\lab5\lab5.5.py}
  \end{minipage}
\end{vvsu_figure}

\begin{vvsu_list}
  \item объявляем функцию help с параметром х;
  \item ищем товара с минимальной ценой с помощью функции min с использванием функции для получния значения по ключу и ложим значение в переменную min1;
  \item находит ключ с максимальным значением в словаре products и ложим в переменную max1;
  \item возвращам переменные min1 и max1;
  \item создаем словарь товаров products;
  \item вызываем функцию help с аргументом products и ложим возвращаемые значения.
\end{vvsu_list}

% Подглава - Задание 6
\subsection{Задание 6}
Имеется список произволных элементов, программа на основе данного списка создает словарь, где каждый элемент списка будт ключем и значением. На рисунке 6 представлен код программы.

\begin{vvsu_figure}{Листинг программы для задания 6}{labs\lab5\lab5.6.py}
  \begin{minipage}{.75\textwidth}
    \lstinputlisting[language=Python,basicstyle=\fontsize{10}{10}\linespread{1}\selectfont\ttfamily]{labs\lab5\lab5.6.py}
  \end{minipage}
\end{vvsu_figure}

\begin{vvsu_list}
  \item создаем списко list1;
  \item объявляем функцию help с параметром х;
  \item создаем пустой словар dict1;
  \item проодимся списком по всем элементам списка x;
  \item добавление в словарь пары ключ-значение, где и ключ и значение - сам элемент;
  \item возврат полученного словаря;
  \item вызов функции help с аргументом list1 и вывод результата.
\end{vvsu_list}

% Подглава - Задание 7
\subsection{Задание 7}
  Программа получает на ввод переменную кортеж, где к английскому слову дан ключ-перевод и с помощью новой созданной функцией eng rus с параметром x- перевод слова и word слово на русском.
  На рисунке 7 представлен код программы.

\begin{vvsu_figure}{Листинг программы для задания 7}{labs\lab5\lab5.7.py}
  \begin{minipage}{.75\textwidth}
    \lstinputlisting[language=Python,basicstyle=\fontsize{10}{10}\linespread{1}\selectfont\ttfamily]{labs\lab5\lab5.7.py}
  \end{minipage}
\end{vvsu_figure}

\begin{vvsu_list}
  \item создается функция eng rus с параметрами x и word;
  \item Цикл по всем парам ключ значение в словаре x;
  \item Проверка, совпадает ли значение словаря с искомым словом;
  \item Если найдено совпадение, возвращается соответствующий английский ключ;
  \item Если совпадений нет, возвращается None;
  \item Создание словаря английский-русский;
  \item Поиск английского слова для русского 'яблоко'.
\end{vvsu_list}


% Подглава - Задание 8
\subsection{Задание 8}
Создаем маленькую программу для игры "Камень-Ножницы-Бумага-Ящерица-Спок". Где пользователь вводит свой выбор, а программа случайным образом выбирает один из вариантов. После чего сравниваются варианты и определяется победитель.
На рисунке 8 представлен код программы.

\begin{vvsu_figure}{Листинг программы для задания 8}{labs\lab5\lab5.8.py}
  \begin{minipage}{.75\textwidth}
    \lstinputlisting[language=Python,basicstyle=\fontsize{10}{10}\linespread{1}\selectfont\ttfamily]{labs\lab5\lab5.8.py}
  \end{minipage}
\end{vvsu_figure}

\begin{vvsu_list}
  \item Импорт модуля для случайного выбора;
  \item Пользователь вводит свой выбор;
  \item Объявление функции игры;
  \item Создание правил: ключ побеждает значения в списке;
  \item Компьютер случайно выбирает вариант;
  \item Вывод выбора компьютера;
  \item Проверка ничьи;
  \item Возврат ничьи;
  \item Проверка, есть ли выбор компьютера в списке побеждаемых вариантов пользователя;
  \item Возврат победы пользователя;
  \item Иначе Компьютер выйграл;
  \item Запуск игры и вывод результата.
\end{vvsu_list}


% Подглава - Задание 9
\subsection{Задание 9}
  Требуется программа, где в списке она находит перую букву и сооздает новый кортеж, где значени, это буква, а ключи, это слова начинающиеся на данную букву. На рисунке 9 представлен код программы.

\begin{vvsu_figure}{Листинг программы для задания 9}{labs\lab5\lab5.9.py}
  \begin{minipage}{.75\textwidth}
    \lstinputlisting[language=Python,basicstyle=\fontsize{10}{10}\linespread{1}\selectfont\ttfamily]{labs\lab5\lab5.9.py}
  \end{minipage}
\end{vvsu_figure}

\begin{vvsu_list}
  \item Создание списка слов;
  \item Объявление функции dict с параметром x (список слов);
  \item Создание множества первых букв всех слов: {'я', 'г', 'б', 'к', 'а'};
  \item Создание словаря, где ключи первые буквы, значения списки слов на эту букву;
  \item Вызов функции и сохранение результата;
  \item Вызов функции и вывод результата.
\end{vvsu_list}

% Подглава - Задание 10
\subsection{Задание 10}
 Требуется программа для подсчета из кортежа средней оценки студентов с оценкой и самим баллом а так же поиска студента с наибольшей средней оценкой и вывести его имя и балл . На рисунке 10 представлен код программы.

\begin{vvsu_figure}{Листинг программы для задания 10}{labs\lab5\lab5.10.py}
  \begin{minipage}{.75\textwidth}
    \lstinputlisting[language=Python,basicstyle=\fontsize{10}{10}\linespread{1}\selectfont\ttfamily]{labs\lab5\lab5.10.py}
  \end{minipage}
\end{vvsu_figure}

\begin{vvsu_list}
  \item Создание списка кортежей с именами студентов и их оценками;
  \item Создание пустого словаря для средних баллов;
  \item Цикл по каждому студенту: извлекает имя и список оценок;
  \item Вычисление среднего балла: сумма оценок делится на количество;
  \item Добавление в словарь: имя студента → средний балл;
  \item Поиск студента с максимальным средним баллом;
  \item Получение максимального среднего балла;
  \item Вывод результата.
\end{vvsu_list}

Спасибо за внимание !

\end{document}