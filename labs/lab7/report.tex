\documentclass[]{vvsu}

\vvsuyear{2025}

%%%%%%%%%%%%%%%%%%%
\usepackage{fontspec}
\usepackage[russian]{babel}
\usepackage{listings}
\usepackage{xcolor}
\usepackage{tabularray}
\usepackage{geometry}
\usepackage{csquotes}

% Стили листингов (встроены, без listing_styles.tex)
\definecolor{codegreen}{rgb}{0,0.6,0}
\definecolor{codegray}{rgb}{0.5,0.5,0.5}
\definecolor{codepurple}{rgb}{0.58,0,0.82}
\definecolor{backcolour}{rgb}{0.95,0.95,0.92}

\lstdefinestyle{python}{
    backgroundcolor=\color{backcolour},
    commentstyle=\color{codegreen},
    keywordstyle=\color{magenta},
    numberstyle=\tiny\color{codegray},
    stringstyle=\color{codepurple},
    basicstyle=\ttfamily\footnotesize,
    breakatwhitespace=false,
    breaklines=true,
    captionpos=b,
    keepspaces=true,
    numbers=left,
    numbersep=5pt,
    showspaces=false,
    showstringspaces=false,
    showtabs=false,
    tabsize=4,
    frame=single
}
\lstset{style=python, language=Python}

\graphicspath{{images/}}
\author{М.В. Кирийчук}
%%%%%%%%%%%%%%%%%%%

\begin{document}

% Шапка
\vvsuhead{%
МИНОБРНАУКИ РОССИИ\\
\vspace{10pt}%
Федеральное государственное бюджетное образовательное учреждение\\
высшего образования\\
<<ВЛАДИВОСТОКСКИЙ ГОСУДАРСТВЕННЫЙ УНИВЕРСИТЕТ>>\\
(ФГБОУ ВО <<ВВГУ>>)\\
\vspace{10pt}%
ИНСТИТУТ ИНФОРМАЦИОННЫХ ТЕХНОЛОГИЙ И АНАЛИЗА ДАННЫХ\\
КАФЕДРА ИНФОРМАЦИОННЫХ ТЕХНОЛОГИЙ И СИСТЕМ%
}

\title{Отчет\\по лабораторной работе №7}
\subtitle{по дисциплине\\<<Информатика и программирование>>}

\member{Студент\\ гр. БИН-25-3}{М.В. Кирийчук}
\member{Ассистент\\ преподавателя}{М.В. Водяницкий}

\maketitle

% Страница "Задание"
\begin{addition}{Задание}
  Выполнить задания и оформить отчет по стандартам ВВГУ.

  \textbf{Задание 1.}  
  Имеется список объектов Фонда с указанием уровня угрозы:
  \begin{verbatim}
objects = [
    ("Containment Cell A", 4),
    ("Archive Vault", 1),
    ("Bio Lab Sector", 3),
    ("Observation Wing", 2)
]
  \end{verbatim}
  Используя \texttt{sorted} и лямбда-выражение, отсортируйте объекты по возрастанию уровня угрозы.

  \textbf{Задание 2.}  
  Дан список сотрудников Фонда с количеством проведенных смен и стоимостью одной смены:
  \begin{verbatim}
staff_shifts = [
    {"name": "Dr. Shaw", "shift_cost": 120, "shifts": 15},
    {"name": "Agent Torres", "shift_cost": 90, "shifts": 22},
    {"name": "Researcher Hall", "shift_cost": 150, "shifts": 10}
]
  \end{verbatim}
  Используя \texttt{map} и лямбда-выражение, создайте список общей стоимости работы каждого сотрудника.  
  Затем найдите максимальную стоимость с помощью \texttt{max}.

  \textbf{Задание 3.}  
  Дан список персонала с уровнем допуска:
  \begin{verbatim}
personnel = [
    {"name": "Dr. Klein", "clearance": 2},
    {"name": "Agent Brooks", "clearance": 4},
    {"name": "Technician Reed", "clearance": 1}
]
  \end{verbatim}
  Используя \texttt{map} и лямбда-выражение, создайте новый список, где каждому сотруднику добавляется категория допуска:
  \begin{vvsu_itemize}
    \item \texttt{"Restricted"} — уровень 1
    \item \texttt{"Confidential"} — уровни 2–3
    \item \texttt{"Top Secret"} — уровень 4 и выше
  \end{vvsu_itemize}
  Результат должен быть списком словарей.

  \textbf{Задание 4.}  
  Дан список зон Фонда с указанием времени активности (в часах):
  \begin{verbatim}
zones = [
    {"zone": "Sector-12", "active_from": 8, "active_to": 18},
    {"zone": "Deep Storage", "active_from": 0, "active_to": 24},
    {"zone": "Research Wing", "active_from": 9, "active_to": 17}
]
  \end{verbatim}
  Используя \texttt{filter} и лямбда-выражение, выберите зоны, которые полностью работают в дневной период (с 8 до 18 включительно).

  \textbf{Задание 5.}  
  Фонд анализирует служебные отчеты. Некоторые отчеты содержат внешние ссылки, которые должны быть удалены перед архивированием (см. полный список в исходном файле).  
  Используя \texttt{filter} и лямбда-выражение:
  \begin{vvsu_itemize}
    \item Отберите отчеты, содержащие ссылки (\texttt{http} или \texttt{https})
    \item Преобразуйте их так, чтобы вместо ссылки отображалось \texttt{[ДАННЫЕ УДАЛЕНЫ]}
  \end{vvsu_itemize}

  \textbf{Задание 6.}  
  Дан список SCP-объектов с указанием их класса содержания:
  \begin{verbatim}
scp_objects = [
    {"scp": "SCP-096", "class": "Euclid"},
    {"scp": "SCP-173", "class": "Euclid"},
    {"scp": "SCP-055", "class": "Keter"},
    {"scp": "SCP-999", "class": "Safe"},
    {"scp": "SCP-3001", "class": "Keter"}
]
  \end{verbatim}
  Используя \texttt{filter} и лямбда-выражение, сформируйте список SCP-объектов, которые требуют усиленных мер содержания.  
  К таким объектам относятся все SCP, \textbf{класс которых не равен} \texttt{"Safe"}.

  \textbf{Задание 7.}  
  Дан список инцидентов с количеством задействованного персонала:
  \begin{verbatim}
incidents = [
    {"id": 101, "staff": 4},
    {"id": 102, "staff": 12},
    {"id": 103, "staff": 7},
    {"id": 104, "staff": 20}
]
  \end{verbatim}
  Используя \texttt{sorted} и лямбда-выражение:
  \begin{vvsu_itemize}
    \item Отсортируйте инциденты по количеству персонала
    \item Оставьте только три наиболее ресурсоемких инцидента
  \end{vvsu_itemize}

  \textbf{Задание 8.}  
  Дан список протоколов безопасности и их уровней критичности:
  \begin{verbatim}
protocols = [
    ("Lockdown", 5),
    ("Evacuation", 4),
    ("Data Wipe", 3),
    ("Routine Scan", 1)
]
  \end{verbatim}
  Используя \texttt{map} и лямбда-выражение, создайте новый список строк вида:  
  \texttt{"Protocol Lockdown - Criticality 5"}.

  \textbf{Задание 9.}  
  Имеется список смен охраны с указанием длительности (в часах):  
  \texttt{shifts = [6, 12, 8, 24, 10, 4]}  
  Используя \texttt{filter} и лямбда-выражение, выберите только те смены, которые:
  \begin{vvsu_itemize}
    \item длятся не менее 8 часов
    \item не превышают 12 часов
  \end{vvsu_itemize}

  \textbf{Задание 10.}  
  Дан список сотрудников с результатами психологической оценки (от 0 до 100):
  \begin{verbatim}
evaluations = [
    {"name": "Agent Cole", "score": 78},
    {"name": "Dr. Weiss", "score": 92},
    {"name": "Technician Moore", "score": 61},
    {"name": "Researcher Lin", "score": 88}
]
  \end{verbatim}
  Используя \texttt{max} и лямбда-выражение, определите сотрудника с наивысшей оценкой.  
  Результатом должно быть имя сотрудника и его балл.
\end{addition}

\toc

\section{Выполнение работы}

\subsection{Задание 1}
Реализована сортировка списка кортежей \texttt{objects}, содержащих название объекта и уровень угрозы. Используется встроенная функция \texttt{sorted()} с ключом-лямбдой \texttt{lambda item: item[1]}, который извлекает второй элемент кортежа (уровень угрозы). Список сортируется по возрастанию. На рисунке~\ref{fig:code_task_1} представлен листинг программы.

\begin{vvsu_figure}{Листинг программы для задания 1}{fig:code_task_1}
  \begin{minipage}{.75\textwidth}
    \lstinputlisting{code/task1.py}
  \end{minipage}
\end{vvsu_figure}

\subsection{Задание 2}
С использованием \texttt{map()} и лямбда-выражения вычисляется общая стоимость работы каждого сотрудника как произведение \texttt{shift\_cost * shifts}. Результат преобразуется в список. Максимальное значение находится через \texttt{max()}. На рисунке~\ref{fig:code_task_2} — листинг.

\begin{vvsu_figure}{Листинг программы для задания 2}{fig:code_task_2}
  \begin{minipage}{.75\textwidth}
    \lstinputlisting{code/task2.py}
  \end{minipage}
\end{vvsu_figure}

\subsection{Задание 3}
Применяется \texttt{map()} с лямбда-функцией, которая для каждого сотрудника добавляет поле \texttt{category} на основе значения \texttt{clearance}. Используется тернарный оператор для определения категории. Результат — новый список словарей. На рисунке~\ref{fig:code_task_3} — листинг.

\begin{vvsu_figure}{Листинг программы для задания 3}{fig:code_task_3}
  \begin{minipage}{.75\textwidth}
    \lstinputlisting{code/task3.py}
  \end{minipage}
\end{vvsu_figure}

\subsection{Задание 4}
С помощью \texttt{filter()} и лямбда-выражения отбираются зоны, у которых \texttt{active\_from >= 8} и \texttt{active\_to <= 18}. Это гарантирует, что зона работает строго в дневное время. На рисунке~\ref{fig:code_task_4} — листинг.

\begin{vvsu_figure}{Листинг программы для задания 4}{fig:code_task_4}
  \begin{minipage}{.75\textwidth}
    \lstinputlisting{code/task4.py}
  \end{minipage}
\end{vvsu_figure}

\subsection{Задание 5}
Сначала \texttt{filter()} выбирает отчёты, содержащие \texttt{http://} или \texttt{https://}. Затем каждый такой отчёт передаётся в функцию \texttt{remove\_all\_links}, которая последовательно заменяет все URL на \texttt{[ДАННЫЕ УДАЛЕНЫ]}. На рисунке~\ref{fig:code_task_5_m} — листинг.

\begin{vvsu_figure}{Листинг программы для задания 5}{fig:code_task_5_m}
  \begin{minipage}{.75\textwidth}
    \lstinputlisting{code/task5m.py}
  \end{minipage}
\end{vvsu_figure}

\subsection{Задание 6}
Используется \texttt{filter()} с лямбда-условием \texttt{obj["class"] != "Safe"}, чтобы оставить только SCP-объекты, требующие усиленных мер. На рисунке~\ref{fig:code_task_6} — листинг.

\begin{vvsu_figure}{Листинг программы для задания 6}{fig:code_task_6}
  \begin{minipage}{.75\textwidth}
    \lstinputlisting{code/task6.py}
  \end{minipage}
\end{vvsu_figure}

\subsection{Задание 7}
Список инцидентов сортируется по убыванию поля \texttt{staff} с помощью \texttt{sorted(..., reverse=True)}. Затем берутся первые три элемента срезом \texttt{[:3]}. На рисунке~\ref{fig:code_task_7} — листинг.

\begin{vvsu_figure}{Листинг программы для задания 7}{fig:code_task_7}
  \begin{minipage}{.75\textwidth}
    \lstinputlisting{code/task7.py}
  \end{minipage}
\end{vvsu_figure}

\subsection{Задание 8}
Применяется \texttt{map()} для форматирования каждой пары \texttt{(protocol, criticality)} в строку требуемого вида с помощью f-строки. На рисунке~\ref{fig:code_task_8} — листинг.

\begin{vvsu_figure}{Листинг программы для задания 8}{fig:code_task_8}
  \begin{minipage}{.75\textwidth}
    \lstinputlisting{code/task8.py}
  \end{minipage}
\end{vvsu_figure}

\subsection{Задание 9}
С помощью \texttt{filter()} и лямбда-выражения \texttt{8 <= x <= 12} отбираются смены, длительность которых находится в заданном диапазоне. На рисунке~\ref{fig:code_task_9} — листинг.

\begin{vvsu_figure}{Листинг программы для задания 9}{fig:code_task_9}
  \begin{minipage}{.75\textwidth}
    \lstinputlisting{code/task9.py}
  \end{minipage}
\end{vvsu_figure}

\subsection{Задание 10}
Функция \texttt{max()} с ключом \texttt{lambda x: x["score"]} находит сотрудника с максимальным баллом. Имя и балл форматируются в строку. На рисунке~\ref{fig:code_task_10} — листинг.

\begin{vvsu_figure}{Листинг программы для задания 10}{fig:code_task_10}
  \begin{minipage}{.75\textwidth}
    \lstinputlisting{code/task10.py}
  \end{minipage}
\end{vvsu_figure}

% Заключение
Таким образом, все десять заданий лабораторной работы №7 выполнены в полном объёме. Программы используют функциональные инструменты Python (\texttt{map}, \texttt{filter}, \texttt{sorted}, \texttt{max}) и лямбда-выражения для обработки структурированных данных. Код корректен, протестирован на примерах из условия и оформлен в соответствии с требованиями СТО ВВГУ.

\end{document}