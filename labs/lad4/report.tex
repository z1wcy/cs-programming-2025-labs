\documentclass[]{vvsu}

\vvsuyear{2025}

%%%%%%%%%%%%%%%%%%%

\usepackage{graphicx} % для изображений
\usepackage{tabularray} % для таблиц
\usepackage{siunitx} % для обозначений (процент, градус)
\usepackage{listings} % для листингов кода

% Список путей, где будут искаться изображения и файлы
\graphicspath{{images/}}

% Автор документа
\author{В.В. Плотникова}

% Настройка стилей для листингов кода
\setmonofont{Consolas}

\makeatletter

\newcommand\language@yaml{yaml}
\expandafter\expandafter\expandafter\lstdefinelanguage
\expandafter{\language@yaml}
{
  keywords={true,false,null,y,n},
  keywordstyle=\color{darkgray},
  basicstyle=\setmainfont{Consolas}\fontsize{8}{8}\linespread{1}\selectfont,
  sensitive=false,
  comment=[l]{\#},
  morecomment=[s]{/*}{*/},
  commentstyle=\color{purple},
  stringstyle=\color{blue},
  moredelim=[l][\color{orange}]{\&},
  moredelim=[l][\color{magenta}]{*},
  moredelim=**[il][\color{red}{:}\color{blue}]{:},
  morestring=[b]',
  morestring=[b]",
  literate =    {---}{{\ProcessThreeDashes}}3
                {>}{{\textcolor{red}\textgreater}}1
                {|}{{\textcolor{red}\textbar}}1
                {\ -\ }{{\ -\ }}3,
}
\lst@AddToHook{EveryLine}{\ifx\lst@language\language@yaml\color{black}\fi}
\makeatother

\lstdefinelanguage{json}{
    basicstyle=\fontsize{8}{8}\linespread{1}\selectfont\ttfamily,
    sensitive=false,
    stringstyle=\color{blue},
    string=[s]{":\ "}{"},
    literate=
        *{0}{{{\color{red}0}}}{1}
         {1}{{{\color{red}1}}}{1}
         {2}{{{\color{red}2}}}{1}
         {3}{{{\color{red}3}}}{1}
         {4}{{{\color{red}4}}}{1}
         {5}{{{\color{red}5}}}{1}
         {6}{{{\color{red}6}}}{1}
         {7}{{{\color{red}7}}}{1}
         {8}{{{\color{red}8}}}{1}
         {9}{{{\color{red}9}}}{1}
}

\definecolor{codegray}{rgb}{0.5,0.5,0.5}
\definecolor{backcolour}{rgb}{0.95,0.95,0.92}
\lstdefinestyle{codestylelst}{
    backgroundcolor=\color{backcolour},
    numberstyle=\color{codegray}\ttfamily,
    breakatwhitespace=false,
    breaklines=true,
    captionpos=b,
    keepspaces=true,
    numbers=left,
    numbersep=5pt,
    showspaces=false,
    showstringspaces=false,
    showtabs=false,
    tabsize=2
}
\lstset{style=codestylelst}


%%%%%%%%%%%%%%%%%%%

\begin{document}

% Шапка
\vvsuhead{\linespread{1}\selectfont{}МИНОБРНАУКИ РОССИИ\\
\vspace{10pt}Федеральное государственное бюджетное образовательное учреждение\\
высшего образования\\
\fontsize{13}{13}\selectfont{}<<ВЛАДИВОСТОКСКИЙ ГОСУДАРСТВЕННЫЙ УНИВЕРСИТЕТ>>\\
(ФГБОУ ВО <<ВВГУ>>)\\
\vspace{10pt}\fontsize{12}{12}\selectfont{}ИНСТИТУТ ИНФОРМАЦИОННЫХ ТЕХНОЛОГИЙ И АНАЛИЗА ДАННЫХ\\
КАФЕДРА ИНФОРМАЦИОННЫХ ТЕХНОЛОГИЙ И СИСТЕМ}

% Название отчета
\title{Отчет\\по лабораторной работе №4}
\subtitle{по дисциплине\\<<Информатика и программирование>>}

% Участники работы
\member{Студент\\ гр. БИН-25-3}{М.В.Кирийчук}
\member{Ассистент\\ преподавателя}{М.В. Водяницкий}

% Вывод титульника
\maketitle

% Задание
\begin{addition}{Задание}
  Выполнить задания и оформить отчет по стандартам ВВГУ.

  \textit{\textbf{Задание 1.}}  
  Написать программу, которая определяет, как будет вести себя кондиционер. Если температура в помещении 20 градусов и выше, то кондиционер выключается, если меньше - включается. Температура должна вводится пользователем с консоли.

  Пример:\\
    Введите температуру: 18\\
    Кондиционер включен

  \textit{\textbf{Задание 2.}}  
  Год делится на четыре сезона: зима, весна, лето и осень. Написать программу, которая запрашивает у пользователя номер месяца и выводит к какому сезону этот месяц относится.

  Пример:\\
    Введите номер месяца: 4\\
    Это весна 

  \textit{\textbf{Задание 3.}}  
  Считается, что один год, прожитый собакой, эквивалентен семи человеческим годам. При этом зачастую не учитывается, что собаки становятся абсолютно взрослыми уже к двум годам. Таким образом, многие предпочитают каждый из первых двух лет жизни собаки приравнивать к 10.5 годам человеческой жизни, а все последующие к 4.
  
  Написать программу, которая будет переводить собачий возраст в человеческий. Программа должна корректно обрабатывать входные данные и выводить соответствующие сообщения об ошибках:

  \begin{vvsu_itemize}
    \item Если вводится не число
    \item Если вводится число меньше 1
    \item Если вводится число большее 22
  \end{vvsu_itemize}

  Пример:\\
    Введите возраст собаки (в годах): 5\\
    Возраст собаки в человеческих годах: 33.0

  Пример:\\
    Введите возраст собаки (в годах): 0\\
    Ошибка: возраст должен быть не меньше 1

  \textit{\textbf{Задание 4.}}  
  Число делиться на 6 только в случае соблюдения двух условий:
  \begin{vvsu_itemize}
    \item Последняя цифра четная
    \item Сумма всех цифр делиться на 3
  \end{vvsu_itemize}
  Написать программу, которая выведет делиться ли введенное число на 6 или нет.

  \textit{\textbf{Задание 5.}}  
  Написать программу, которая будет проверять пароль на надежность. Пароль считается надежным, если его длина не менее 8 символов и если он содержит:

  \begin{vvsu_itemize}
    \item Заглавные буквы латиницы
    \item Строчные буквы латиницы
    \item Числа
    \item Специальные знаки
  \end{vvsu_itemize}

  В случае, если пароль не проходит по одному из условий, необходимо сообщить пользователю каким именно условиям он не удовлетворяет.

  Пример:\\
    Введите пароль: qwerty\\
    Пароль ненадежный: отсутствуют заглавные буквы, числа и специальные символы

  \textit{\textbf{Задание 6.}}  
  Написать программу, которая определяет, является ли введенный пользователем год високосным. Год считается високосным, если он делится на 4, но не делится на 100, либо если он делится на 400.

  Пример:\\
    Введите год: 2024\\
    2024 - високосный год

  \textit{\textbf{Задание 7.}}  
  Написать программу, которая запрашивает у пользователя три числа и выводит на экран наименьшее из них. При решении нельзя использовать встроенные функции min() и max().

  Пример:\\
    Введите три числа: 8 3 5\\
    Наименьшее число: 3

  \textit{\textbf{Задание 8.}}  
  В магазине проводится акция. Акция работает по следующим правилам:

  \begin{vvsu_itemize}
    \item Сумма < 1000 => скидка - 0\%
    \item Сумма < 5000 => скидка - 5\%
    \item Сумма < 10000 => скидка - 10\%
    \item Сумма > 10000 => скидка - 15\%
  \end{vvsu_itemize}

  Напишите программу, которая запрашивает сумму покупки и выводит размер скидки и итоговую сумму к оплате.
  
  Пример:\\
    Введите сумму покупки: 7500\\
    Ваша скидка: 10%\\
    К оплате : 6750.0
  
  \textit{\textbf{Задание 9.}}  
  Написать программу, которая определяет время суток по введенному часу (целое число от 0 до 23).

  \begin{vvsu_itemize}
    \item С 0 до 5 часов - ночь
    \item С 6 до 11 часов - утро
    \item С 12 до 17 часов - день
    \item С 18 до 23 часов - вечер
  \end{vvsu_itemize}
  
  Пример:\\
    Введите час (0–23): 20\\
    Сейчас вечер

  \textit{\textbf{Задание 10.}}  
  Написать программу, которая определяет, является ли введенное число простым. Число называется простым, если оно больше 1 и делится только на 1 и само себя. Программа должна корректно обрабатывать некорректный ввод и выводить соответствующие сообщения об ошибках.
  
  Пример:\\
    Введите число: 17\\
    17 - простое число
\end{addition}


% Содержание
\newcommand{\toc} {
  {\centering\fontsize{14}{14}\linespread{1}\setmainfont{Arial}\selectfont Содержание\vspace{8pt}\\}
  \@starttoc{toc}%
  \clearpage
}

% Глава - Выполнение работы
\section{Выполнение работы}

% Подглава - Задание 1
\subsection{Задание 1}

Объявляется функция с именем air conditioner без параметров, функция input выводит текстовое приглашение и ожидает ввод от пользователя. Введенная строка преобразуется в вещественное число с помощью float, полученное числовое значение присваивается переменной temp. Выполняется сравнение значения глобальной переменной temp с числом 20, оператор >= провеляет, является ли температура больше или равной 20. При истинности условия выполняется первый блок кода с сообщением о выключении, при ложности условия выполняется альтернативный блок кода с сообщением о включении. На рисунке \ref{fig:code_task_1} представлен код программы.

\begin{vvsu_figure}{Листинг программы для задания 1}{fig:code_task_1}
  \begin{minipage}{.75\textwidth}
    \lstinputlisting[language=Python,basicstyle=\fontsize{10}{10}\linespread{1}\selectfont\ttfamily]{code/task1.py}
  \end{minipage}
\end{vvsu_figure}

% Подглава - Задание 2
\subsection{Задание 2}

 Объявляется функция season с параметром num, num ожидает получить целое число от 1 до 12. Функция input запрашивает ввод номера месяца, int преобразует строку в целое число. Переменная num сохраняет введенное значение. Функция season вызывается с передачей параметра num. Проверяется корректность переданного номера месяца, условие num < 1 отсекает числа меньше 1, условие num > 12 отсекает числа больше 12.
При нарушении любого условия выводится сообщение об ошибке.
Проверяется принадлежность к зимним месяцам: используется список [1, 2, 12] для идентификации зимы; оператор in проверяет вхождение значения в список.
Идентично для остальных.
 Оставшиеся месяцы (сентябрь, октябрь, ноябрь) классифицируются как осенние.
 На рисунке \ref{fig:code_task_2} представлен код программы.

\begin{vvsu_figure}{Листинг программы для задания 2}{fig:code_task_2}
  \begin{minipage}{.75\textwidth}
    \lstinputlisting[language=Python,basicstyle=\fontsize{10}{10}\linespread{1}\selectfont\ttfamily]{code/task2.py}
  \end{minipage}
\end{vvsu_figure}

% Подглава - Задание 3
\subsection{Задание 3}

Объявляется функция без параметров: def dog to human age:
Запрашиваем возраст собаки у пользователя. Далее обрабатываем значение - возраст не может быть меньше 1 и больше 22, при нарушении выводится ошибка и функция завершается.
Далее конвертируем начальное значение в человеческий эквивалент следующим образом: если возраст собаки 2 и более, то выводим (10.5*возраст) собаки, в ином случае (4.5*(возраст собаки-2)).
 На рисунке \ref{fig:code_task_3} представлен код программы.

\begin{vvsu_figure}{Листинг программы для задания 3}{fig:code_task_3}
  \begin{minipage}{.75\textwidth}
    \lstinputlisting[language=Python,basicstyle=\fontsize{10}{10}\linespread{1}\selectfont\ttfamily]{code/task3.py}
  \end{minipage}
\end{vvsu_figure}

% Подглава - Задание 4
\subsection{Задание 4}

Сперва просим пользователя ввести интересующее его число, далее записываем в переменную num. Затем проверим число на четность. После проверяем делится ли сумма цифр числа на 3 без остатка. В случае соблюдения всех условий, выводим в консоль - число делится на 6, в ином случае - не делится. На рисунке \ref{fig:code_task_4} представлен код решения.

\begin{vvsu_figure}{Листинг программы для задания 4}{fig:code_task_4}
  \begin{minipage}{.75\textwidth}
    \lstinputlisting[language=Python,basicstyle=\fontsize{10}{10}\linespread{1}\selectfont\ttfamily]{code/task4.py}
  \end{minipage}
\end{vvsu_figure}


% Подглава - Задание 5
\subsection{Задание 5}

Запрашиваем ввод пароля от пользователя и записываем в переменную password. Создаем словарь с условиями проверки пароля: 
Пароль не менее 8 символов; наличие заглавных букв латиницы (any() возвращает True, если хотя бы один символ удовлетворяет условию, c.isupper() проверяет, является ли символ заглавной буквой); наличие строчных букв латиницы; наличие цифр (c.isdigit() проверяет, является ли символ цифрой); проверка наличия специальных символов (not c.isalnum() проверяет, НЕ является ли символ буквой или цифрой). Проверяем каждое условие с помощью цикла (items() возвращает пары (ключ, значение) из словаря).
 Если хотя бы одно условие не выполнено, общий результат становится False – пароль ненадежный. На рисунке \ref{fig:code_task_5} представлен код программы.

\begin{vvsu_figure}{Листинг программы для задания 5}{fig:code_task_5}
  \begin{minipage}{.75\textwidth}
    \lstinputlisting[language=Python,basicstyle=\fontsize{10}{10}\linespread{1}\selectfont\ttfamily]{code/task5.py}
  \end{minipage}
\end{vvsu_figure}

% Подглава - Задание 6
\subsection{Задание 6}

Год является високосным, если делится на 4 без остатка и не делится на 100 или делится на 400. Проверяем делимость. Оператор and выполняется первым, or выполняется вторым. На рисунке \ref{fig:code_task_6} представлен код программы.

\begin{vvsu_figure}{Листинг программы для задания 6}{fig:code_task_6}
  \begin{minipage}{.75\textwidth}
    \lstinputlisting[language=Python,basicstyle=\fontsize{10}{10}\linespread{1}\selectfont\ttfamily]{code/task6.py}
  \end{minipage}
\end{vvsu_figure}

% Подглава - Задание 7
\subsection{Задание 7}

Сперва запрашиваем три числа у пользователя. Далее ищем минимальные значения через условные операторы.
Первая проверка:
Проверяет, является ли a меньше или равным b И a меньше или равным c.
Если оба условия истинны, a - наименьшее число.
Вторая проверка:
Выполняется, если первое условие ложно.
Проверяет, является ли b меньше или равным a И b меньше или равным c.
Если истинно, b - наименьшее число.
Третий случай: else:
Срабатывает, если оба предыдущих условия ложны.
Значит, c - наименьшее число
Найденное наименьшее число выводится с использованием f-строки. На рисунке \ref{fig:code_task_7} представлен код программы.

\begin{vvsu_figure}{Листинг программы для задания 7}{fig:code_task_7}
  \begin{minipage}{.75\textwidth}
    \lstinputlisting[language=Python,basicstyle=\fontsize{10}{10}\linespread{1}\selectfont\ttfamily]{code/task7.py}
  \end{minipage}
\end{vvsu_figure}

% Подглава - Задание 8
\subsection{Задание 8}

Получаем исходную сумму покупки от пользователя для дальнейших расчетов. - преобразует введенную строку в число с плавающей точкой.
Размер скидки зависит от суммы покупки: сумма до 1000р. скидка 0 процентов, сумма от 1000 до 5000 рублей включительно, скидка: 5 процентов, сумма от 5001 до 10000 рублей включительно, скидка: 10 процентов; сумма свыше 10000 рублей, скидка: 15 процентов. Затем по формуле рассчитывается и выводится итоговая сумма к оплате с учетом скидки. На рисунке \ref{fig:code_task_8} представлен код программы.

\begin{vvsu_figure}{Листинг программы для задания 8}{fig:code_task_8}
  \begin{minipage}{.75\textwidth}
    \lstinputlisting[language=Python,basicstyle=\fontsize{10}{10}\linespread{1}\selectfont\ttfamily]{code/task8.py}
  \end{minipage}
\end{vvsu_figure}

% Подглава - Задание 9
\subsection{Задание 9}

Создаем функцию time с одним параметром num, параметр num представляет час суток (0-23). После проверяем валидность входных данных:
0 > num < 23; часы не могут быть отрицательными/ в сутках не более 23 часов.
При невалидном вводе выводится сообщение об ошибке.
Если данные коректны, то проверяется принадлежность к времени суток с помощью оператора in и заранее определенных списков с часами каждого из времени суток. На рисунке \ref{fig:code_task_9} представлен код программы.

\begin{vvsu_figure}{Листинг программы для задания 9}{fig:code_task_9}
  \begin{minipage}{.75\textwidth}
    \lstinputlisting[language=Python,basicstyle=\fontsize{10}{10}\linespread{1}\selectfont\ttfamily]{code/task9.py}
  \end{minipage}
\end{vvsu_figure}

% Подглава - Задание 10
\subsection{Задание 10}

Получаем число от пользователя с помощью input(), int() преобразовывает строку в целое число. При вводе нечисловых данных выводим ошибку.
Проводим проверки: простое число должно быть больше 1; число 2; число делится на 2 без остатка. Проверяем нечетные числа: число простое до тех пор, пока не найден делитель. Устанавливаем флаг lit = False и выходим из цикла.
Обработка ошибок: ValueError - некорректный ввод (буквы, символы), Exception - любые другие непредвиденные ошибки.
 На рисунке \ref{fig:code_task_10} представлен код программы.

\begin{vvsu_figure}{Листинг программы для задания 10}{fig:code_task_10}
  \begin{minipage}{.75\textwidth}
    \lstinputlisting[language=Python,basicstyle=\fontsize{10}{10}\linespread{1}\selectfont\ttfamily]{code/task10.py}
  \end{minipage}
\end{vvsu_figure}

Тут должен быть логический вывод, спасибо за внимание !

\end{document}