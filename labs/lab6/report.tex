\documentclass[]{vvsu}

\vvsuyear{2025}

%%%%%%%%%%%%%%%%%%%

\usepackage{graphicx} % для изображений
\usepackage{tabularray} % для таблиц
\usepackage{siunitx} % для обозначений (процент, градус)
\usepackage{listings} % для листингов кода

% Список путей, где будут искаться изображения и файлы
\graphicspath{{images/}}

% Автор документа
\author{М.В. Кирийчук}

% Настройка стилей для листингов кода
\setmonofont{Consolas}

\makeatletter

\newcommand\language@yaml{yaml}
\expandafter\expandafter\expandafter\lstdefinelanguage
\expandafter{\language@yaml}
{
  keywords={true,false,null,y,n},
  keywordstyle=\color{darkgray},
  basicstyle=\setmainfont{Consolas}\fontsize{8}{8}\linespread{1}\selectfont,
  sensitive=false,
  comment=[l]{\#},
  morecomment=[s]{/*}{*/},
  commentstyle=\color{purple},
  stringstyle=\color{blue},
  moredelim=[l][\color{orange}]{\&},
  moredelim=[l][\color{magenta}]{*},
  moredelim=**[il][\color{red}{:}\color{blue}]{:},
  morestring=[b]',
  morestring=[b]",
  literate =    {---}{{\ProcessThreeDashes}}3
                {>}{{\textcolor{red}\textgreater}}1
                {|}{{\textcolor{red}\textbar}}1
                {\ -\ }{{\ -\ }}3,
}
\lst@AddToHook{EveryLine}{\ifx\lst@language\language@yaml\color{black}\fi}
\makeatother

\lstdefinelanguage{json}{
    basicstyle=\fontsize{8}{8}\linespread{1}\selectfont\ttfamily,
    sensitive=false,
    stringstyle=\color{blue},
    string=[s]{":\ "}{"},
    literate=
        *{0}{{{\color{red}0}}}{1}
         {1}{{{\color{red}1}}}{1}
         {2}{{{\color{red}2}}}{1}
         {3}{{{\color{red}3}}}{1}
         {4}{{{\color{red}4}}}{1}
         {5}{{{\color{red}5}}}{1}
         {6}{{{\color{red}6}}}{1}
         {7}{{{\color{red}7}}}{1}
         {8}{{{\color{red}8}}}{1}
         {9}{{{\color{red}9}}}{1}
}

\definecolor{codegray}{rgb}{0.5,0.5,0.5}
\definecolor{backcolour}{rgb}{0.95,0.95,0.92}
\lstdefinestyle{codestylelst}{
    backgroundcolor=\color{backcolour},
    numberstyle=\color{codegray}\ttfamily,
    breakatwhitespace=false,
    breaklines=true,
    captionpos=b,
    keepspaces=true,
    numbers=left,
    numbersep=5pt,
    showspaces=false,
    showstringspaces=false,
    showtabs=false,
    tabsize=2
}
\lstset{style=codestylelst}


%%%%%%%%%%%%%%%%%%%

\begin{document}

% Шапка
\vvsuhead{\linespread{1}\selectfont{}МИНОБРНАУКИ РОССИИ\\
\vspace{10pt}Федеральное государственное бюджетное образовательное учреждение\\
высшего образования\\
\fontsize{13}{13}\selectfont{}<<ВЛАДИВОСТОКСКИЙ ГОСУДАРСТВЕННЫЙ УНИВЕРСИТЕТ>>\\
(ФГБОУ ВО <<ВВГУ>>)\\
\vspace{10pt}\fontsize{12}{12}\selectfont{}ИНСТИТУТ ИНФОРМАЦИОННЫХ ТЕХНОЛОГИЙ И АНАЛИЗА ДАННЫХ\\
КАФЕДРА ИНФОРМАЦИОННЫХ ТЕХНОЛОГИЙ И СИСТЕМ}

% Название отчета
\title{Отчет\\по лабораторной работе №6}
\subtitle{по дисциплине\\<<Информатика и программирование>>}

% Участники работы
\member{Студент\\ гр. БИН-25-3}{М.В. Кирийчук}
\member{Ассистент\\ преподавателя}{М.В. Водяницкий}

% Вывод титульника
\maketitle

% Задание
\begin{addition}{Задание}
  Выполнить задания и оформить отчет по стандартам ВВГУ.

  \textit{\textbf{Задание 1.}}  
  Написать функцию, которая конвертирует время из одной величины в другую.  
  На вход подаются:  
  \begin{vvsu_itemize}
    \item число (величина времени)
    \item исходная единица измерения
    \item единица измерения, в которую нужно перевести
  \end{vvsu_itemize}
  Функция должна вернуть конвертированное значение.

  Пример:\\
    Введите время для конвертации: 4 h m\\
    Результат: 240m

  \textit{\textbf{Задание 2.}}  
  Пользователь делает вклад в банке в размере \texttt{a} рублей сроком на \texttt{n} лет. Процент по вкладу зависит от суммы и срока.

  \textbf{Зависимость от суммы:}
  \begin{vvsu_itemize}
    \item каждые 10\,000 рублей увеличивают ставку на 0.3\%
    \item суммарное увеличение не может превышать 5\%
    \item минимальный вклад — 30\,000 рублей
  \end{vvsu_itemize}

  \textbf{Зависимость от срока:}
  \begin{vvsu_itemize}
    \item первые 3 года — 3\%
    \item от 4 до 6 лет — 5\%
    \item более 6 лет — 2\%
  \end{vvsu_itemize}

  Необходимо написать функцию, которая рассчитывает \textbf{прибыль пользователя без учета первоначально вложенной суммы}. Используется сложный процент.

  Пример:\\
    Ввод: 30000 3\\
    Прибыль: 3648.67 руб.

  \textit{\textbf{Задание 3.}}  
  Написать функцию для вывода всех простых чисел в заданном диапазоне. Нужно учитывать некорректные данные (например, начало больше конца или диапазон без простых чисел).

  На вход подаются два числа: начало и конец диапазона (включительно).  
  На выходе — список всех простых чисел или сообщение об ошибке.

  Пример:\\
    Начало диапазона: 1\\
    Конец диапазона: 10\\
    2 3 5 7

  Пример:\\
    Начало диапазона: 0\\
    Конец диапазона: 1\\
    Error!

  \textit{\textbf{Задание 4.}}  
  Реализовать функцию сложения двух матриц.  
  При сложении двух матриц получается новая матрица того же размера, где каждый элемент — это сумма элементов с тем же индексом из двух исходных матриц.

  \textbf{Ограничения:}
  \begin{vvsu_itemize}
    \item складывать можно только матрицы одинакового размера
    \item размер матрицы должен быть строго больше 2 (например, 3×3, 4×4 и т.д.)
    \item при нарушении условий нужно вывести сообщение об ошибке
  \end{vvsu_itemize}

  На вход подаются:
  \begin{vvsu_list}
    \item размер матрицы \texttt{n} (для квадратной матрицы \texttt{n × n})
    \item элементы первой матрицы (по строкам, через пробел)
    \item элементы второй матрицы (в таком же формате)
  \end{vvsu_list}

  Результат — новая матрица (в том же формате), либо сообщение об ошибке.

  \textit{\textbf{Задание 5.}}  
  Написать функцию, которая определяет, является ли строка палиндромом.  
  Палиндром — это строка, которая читается одинаково слева направо и справа налево (без учета пробелов, регистра и знаков препинания).

  На вход подается строка.  
  На выходе:
  \begin{vvsu_itemize}
    \item \texttt{Да}, если это палиндром
    \item \texttt{Нет}, если это не палиндром
  \end{vvsu_itemize}

  Пример:\\
    А роза упала на лапу Азора\\
    Да

  Пример:\\
    Алфавитный порядок\\
    Нет
\end{addition}

% Содержание
\toc

% Глава - Выполнение работы
\section{Выполнение работы}

% Подглава - Задание 1
\subsection{Задание 1}

Реализована функция \texttt{time\_convert()}, предназначенная для конвертации времени между различными единицами измерения: секунды (\texttt{s}), минуты (\texttt{m}), часы (\texttt{h}) и дни (\texttt{d}). Внутри функции определён словарь \texttt{time}, содержащий ключи-синонимы единиц времени и соответствующие значения в секундах. Пользователь вводит строку в формате «число исходная\_единица целевая\_единица». Программа разбивает ввод на три компонента, преобразует значение в вещественное число и приводит единицы к нижнему регистру. Если обе единицы распознаны, выполняется перевод по формуле:  
\[
\text{результат} = \frac{\text{значение} \cdot \text{коэффициент\_исходной}}{\text{коэффициент\_целевой}}.
\]  
Результат выводится как целое число (если возможно) или с двумя знаками после запятой. Обрабатываются ошибки формата и неизвестных единиц. На рисунке \ref{fig:code_task_1} представлен код программы.

\begin{vvsu_figure}{Листинг программы для задания 1}{fig:code_task_1}
  \begin{minipage}{.75\textwidth}
    \lstinputlisting[language=Python,basicstyle=\fontsize{10}{10}\linespread{1}\selectfont\ttfamily]{code/task1.py}
  \end{minipage}
\end{vvsu_figure}

% Подглава - Задание 2
\subsection{Задание 2}

Реализована функция \texttt{calculate\_profit(a, n)}, вычисляющая прибыль по банковскому вкладу с учётом сложного процента. Сначала проверяется, что сумма вклада не менее 30\,000 руб. Ставка по сроку определяется условной конструкцией: 3\% (≤3 лет), 5\% (4–6 лет), 2\% (>6 лет). Дополнительная ставка за сумму рассчитывается как \texttt{min((a // 10000) * 0.003, 0.05)}. Итоговая прибыль вычисляется по формуле сложного процента:  
\[
\text{прибыль} = a \cdot (1 + r_{\text{срок}} + r_{\text{сумма}})^n - a.
\]  
Результат округляется до двух знаков. Обёртка \texttt{main\_compact()} обеспечивает удобный ввод в цикле. На рисунке \ref{fig:code_task_2} представлен код программы.

\begin{vvsu_figure}{Листинг программы для задания 2}{fig:code_task_2}
  \begin{minipage}{.75\textwidth}
    \lstinputlisting[language=Python,basicstyle=\fontsize{10}{10}\linespread{1}\selectfont\ttfamily]{code/task2.py}
  \end{minipage}
\end{vvsu_figure}

% Подглава - Задание 3
\subsection{Задание 3}

Реализована функция \texttt{primes(a, b)}, возвращающая список всех простых чисел в диапазоне \texttt{[a, b]}. Перебор начинается с \texttt{max(2, a)}, так как простые числа начинаются с 2. Для каждого числа проверяется делимость нацело на числа от 2 до \(\sqrt{\text{num}}\). Если делителей нет, число добавляется в список. Если список пуст, возвращается строка \texttt{"Error!"}. Ввод границ диапазона осуществляется с консоли. На рисунке \ref{fig:code_task_3} представлен код программы.

\begin{vvsu_figure}{Листинг программы для задания 3}{fig:code_task_3}
  \begin{minipage}{.75\textwidth}
    \lstinputlisting[language=Python,basicstyle=\fontsize{10}{10}\linespread{1}\selectfont\ttfamily]{code/task3.py}
  \end{minipage}
\end{vvsu_figure}

% Подглава - Задание 4
\subsection{Задание 4}

Реализована функция \texttt{add\_matrices()}, выполняющая сложение двух квадратных матриц размера \texttt{n × n}. Сначала запрашивается размер \texttt{n}; если \texttt{n ≤ 2}, выводится ошибка. Затем пользователь вводит две матрицы построчно. Каждая строка преобразуется в список целых чисел. После ввода выполняется поэлементное сложение и вывод результата. Любая ошибка ввода (например, недостаточное количество чисел) перехватывается общим исключением, и выводится \texttt{Error!}. На рисунке \ref{fig:code_task_4} представлен код программы.

\begin{vvsu_figure}{Листинг программы для задания 4}{fig:code_task_4}
  \begin{minipage}{.75\textwidth}
    \lstinputlisting[language=Python,basicstyle=\fontsize{10}{10}\linespread{1}\selectfont\ttfamily]{code/task4.py}
  \end{minipage}
\end{vvsu_figure}

% Подглава - Задание 5
\subsection{Задание 5}

Реализована проверка строки на палиндром. Строка очищается от всех не-буквенно-цифровых символов с помощью генератора \texttt{c.lower() for c in s if c.isalnum()}, после чего сравнивается с её обратным порядком (\texttt{cleaned[::-1]}). Если строки совпадают, выводится \texttt{Да}, иначе — \texttt{Нет}. Программа корректно обрабатывает смешанный регистр, пробелы и пунктуацию. На рисунке \ref{fig:code_task_5} представлен код программы.

\begin{vvsu_figure}{Листинг программы для задания 5}{fig:code_task_5}
  \begin{minipage}{.75\textwidth}
    \lstinputlisting[language=Python,basicstyle=\fontsize{10}{10}\linespread{1}\selectfont\ttfamily]{code/task5.py}
  \end{minipage}
\end{vvsu_figure}

% Заключение
Таким образом, все пять заданий лабораторной работы №6 выполнены в полном объёме. Программы корректно обрабатывают ошибки ввода, соответствуют постановке задач и протестированы на примерах из условия. Отчёт оформлен в соответствии с требованиями СТО ВВГУ.

\end{document}