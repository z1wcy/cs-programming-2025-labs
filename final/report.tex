\documentclass[]{vvsu}

\vvsuyear{2025}

%%%%%%%%%%%%%%%%%%%
\usepackage{fontspec}
\usepackage[russian]{babel}
\usepackage{listings}
\usepackage{xcolor}
\usepackage{tabularray}
\usepackage{geometry}
\usepackage{csquotes}

% Стили листингов (встроены, без listing_styles.tex)
\definecolor{codegreen}{rgb}{0,0.6,0}
\definecolor{codegray}{rgb}{0.5,0.5,0.5}
\definecolor{codepurple}{rgb}{0.58,0,0.82}
\definecolor{backcolour}{rgb}{0.95,0.95,0.92}

\lstdefinestyle{python}{
    backgroundcolor=\color{backcolour},
    commentstyle=\color{codegreen},
    keywordstyle=\color{magenta},
    numberstyle=\tiny\color{codegray},
    stringstyle=\color{codepurple},
    basicstyle=\ttfamily\footnotesize,
    breakatwhitespace=false,
    breaklines=true,
    captionpos=b,
    keepspaces=true,
    numbers=left,
    numbersep=5pt,
    showspaces=false,
    showstringspaces=false,
    showtabs=false,
    tabsize=4,
    frame=single
}
\lstset{style=python, language=Python}

\graphicspath{{images/}}
\author{М.В. Кирийчук}
%%%%%%%%%%%%%%%%%%%

\begin{document}

% Шапка
\vvsuhead{%
МИНОБРНАУКИ РОССИИ\\
\vspace{10pt}%
Федеральное государственное бюджетное образовательное учреждение\\
высшего образования\\
<<ВЛАДИВОСТОКСКИЙ ГОСУДАРСТВЕННЫЙ УНИВЕРСИТЕТ>>\\
(ФГБОУ ВО <<ВВГУ>>)\\
\vspace{10pt}%
ИНСТИТУТ ИНФОРМАЦИОННЫХ ТЕХНОЛОГИЙ И АНАЛИЗА ДАННЫХ\\
КАФЕДРА ИНФОРМАЦИОННЫХ ТЕХНОЛОГИЙ И СИСТЕМ%
}

\title{Отчет\\по лабораторной работе №7}
\subtitle{по дисциплине\\<<Информатика и программирование>>}

\member{Студент\\ гр. БИН-25-3}{М.В. Кирийчук}ё
\member{Ассистент\\ преподавателя}{М.В. Водяницкий}

\maketitle



\toc

\section{Введение}


Реализована сортировка списка кортежей \texttt{objects}, содержащих название объекта и уровень угрозы. Используется встроенная функция \texttt{sorted()} с ключом-лямбдой \texttt{lambda item: item[1]}, который извлекает второй элемент кортежа (уровень угрозы). Список сортируется по возрастанию. На рисунке~\ref{fig:code_task_1} представлен листинг программы.

\begin{vvsu_figure}{Листинг программы для задания 1}{fig:code_task_1}
  \begin{minipage}{.75\textwidth}
    \lstinputlisting{code/task1.py}
  \end{minipage}
\end{vvsu_figure}

\subsection{Задание 2}
С использованием \texttt{map()} и лямбда-выражения вычисляется общая стоимость работы каждого сотрудника как произведение \texttt{shift\_cost * shifts}. Результат преобразуется в список. Максимальное значение находится через \texttt{max()}. На рисунке~\ref{fig:code_task_2} — листинг.

\begin{vvsu_figure}{Листинг программы для задания 2}{fig:code_task_2}
  \begin{minipage}{.75\textwidth}
    \lstinputlisting{code/task2.py}
  \end{minipage}
\end{vvsu_figure}

\subsection{Задание 3}
Применяется \texttt{map()} с лямбда-функцией, которая для каждого сотрудника добавляет поле \texttt{category} на основе значения \texttt{clearance}. Используется тернарный оператор для определения категории. Результат — новый список словарей. На рисунке~\ref{fig:code_task_3} — листинг.

\begin{vvsu_figure}{Листинг программы для задания 3}{fig:code_task_3}
  \begin{minipage}{.75\textwidth}
    \lstinputlisting{code/task3.py}
  \end{minipage}
\end{vvsu_figure}

\subsection{Задание 4}
С помощью \texttt{filter()} и лямбда-выражения отбираются зоны, у которых \texttt{active\_from >= 8} и \texttt{active\_to <= 18}. Это гарантирует, что зона работает строго в дневное время. На рисунке~\ref{fig:code_task_4} — листинг.

\begin{vvsu_figure}{Листинг программы для задания 4}{fig:code_task_4}
  \begin{minipage}{.75\textwidth}
    \lstinputlisting{code/task4.py}
  \end{minipage}
\end{vvsu_figure}

\subsection{Задание 5}
Сначала \texttt{filter()} выбирает отчёты, содержащие \texttt{http://} или \texttt{https://}. Затем каждый такой отчёт передаётся в функцию \texttt{remove\_all\_links}, которая последовательно заменяет все URL на \texttt{[ДАННЫЕ УДАЛЕНЫ]}. На рисунке~\ref{fig:code_task_5} — листинг.

\begin{vvsu_figure}{Листинг программы для задания 5}{fig:code_task_5}
  \begin{minipage}{.75\textwidth}
    \lstinputlisting{code/task5.py}
  \end{minipage}
\end{vvsu_figure}

\subsection{Задание 6}
Используется \texttt{filter()} с лямбда-условием \texttt{obj["class"] != "Safe"}, чтобы оставить только SCP-объекты, требующие усиленных мер. На рисунке~\ref{fig:code_task_6} — листинг.

\begin{vvsu_figure}{Листинг программы для задания 6}{fig:code_task_6}
  \begin{minipage}{.75\textwidth}
    \lstinputlisting{code/task6.py}
  \end{minipage}
\end{vvsu_figure}

\subsection{Задание 7}
Список инцидентов сортируется по убыванию поля \texttt{staff} с помощью \texttt{sorted(..., reverse=True)}. Затем берутся первые три элемента срезом \texttt{[:3]}. На рисунке~\ref{fig:code_task_7} — листинг.

\begin{vvsu_figure}{Листинг программы для задания 7}{fig:code_task_7}
  \begin{minipage}{.75\textwidth}
    \lstinputlisting{code/task7.py}
  \end{minipage}
\end{vvsu_figure}

\subsection{Задание 8}
Применяется \texttt{map()} для форматирования каждой пары \texttt{(protocol, criticality)} в строку требуемого вида с помощью f-строки. На рисунке~\ref{fig:code_task_8} — листинг.

\begin{vvsu_figure}{Листинг программы для задания 8}{fig:code_task_8}
  \begin{minipage}{.75\textwidth}
    \lstinputlisting{code/task8.py}
  \end{minipage}
\end{vvsu_figure}

\subsection{Задание 9}
С помощью \texttt{filter()} и лямбда-выражения \texttt{8 <= x <= 12} отбираются смены, длительность которых находится в заданном диапазоне. На рисунке~\ref{fig:code_task_9} — листинг.

\begin{vvsu_figure}{Листинг программы для задания 9}{fig:code_task_9}
  \begin{minipage}{.75\textwidth}
    \lstinputlisting{code/task9.py}
  \end{minipage}
\end{vvsu_figure}

\subsection{Задание 10}
Функция \texttt{max()} с ключом \texttt{lambda x: x["score"]} находит сотрудника с максимальным баллом. Имя и балл форматируются в строку. На рисунке~\ref{fig:code_task_10} — листинг.

\begin{vvsu_figure}{Листинг программы для задания 10}{fig:code_task_10}
  \begin{minipage}{.75\textwidth}
    \lstinputlisting{code/task10.py}
  \end{minipage}
\end{vvsu_figure}

% Заключение
Таким образом, все десять заданий лабораторной работы №7 выполнены в полном объёме. Программы используют функциональные инструменты Python (\texttt{map}, \texttt{filter}, \texttt{sorted}, \texttt{max}) и лямбда-выражения для обработки структурированных данных. Код корректен, протестирован на примерах из условия и оформлен в соответствии с требованиями СТО ВВГУ.

\end{document}